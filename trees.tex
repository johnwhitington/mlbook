\documentclass[]{book}
\usepackage{tikz}
\usepackage{tikz-qtree}

\begin{document}
\Tree [.$+$ 1 [.$\times$ 2 3 ] ]

%\Tree [.$+$ 1 [.$\times$ 2 3 ] ] \hspace{4mm}$\longrightarrow$\hspace{4mm} \Tree [.$+$ 1 6 ] \hspace{4mm}$\longrightarrow$\hspace{4mm} \Tree [.7 ] 


%\Tree [.1 \phantom{3} \phantom{3} ]\hspace{4mm}is written as\hspace{4mm}\texttt{Br (1, Lf, Lf)}

%\Tree [.2 [.1 \phantom{3} \phantom{3} ] \phantom{3} ]\hspace{4mm}is written as\hspace{4mm}\texttt{Br (2, Br (1, Lf, Lf), Lf)}

%\Tree [.2 [.1 \phantom{3} \phantom{3} ] [.4 \phantom{3} \phantom{3} ] ]\hspace{4mm} is written as\hspace{4mm} \texttt{Br (2, Br (1, Lf, Lf), Br (4, Lf, Lf))}

%\Tree [.\texttt{(3, "three")} [.\texttt{(1, "one")} \phantom{a} [.\texttt{(2, "two")} \phantom{a} \phantom{a} ] ] [.\texttt{(4, "four")} \phantom{a} \phantom{a} ] ]

%\Tree [.\texttt{(3, "three")} [.\texttt{(1, "one")} [.\texttt{(0, "zero")} \phantom{a} \phantom{a} ] [.\texttt{(2, "two")} \phantom{a} \phantom{a} ] ] [.\texttt{(4, "four")} \phantom{a} \phantom{a} ] ]
\end{document}

